\documentclass[conference]{IEEEtran}

\usepackage{graphicx}
\usepackage{amsmath}
\usepackage{amsfonts}
\usepackage{amssymb}

\begin{document}

\title{Software Requirements Specification for Ecometer}

\author{\IEEEauthorblockN{EcoMeter}
\IEEEauthorblockA{
EcoMeter@gmail.com}
}

\maketitle

\section{Introduction}

\subsection{Purpose}
The purpose of the web application Ecometer is to measure energy efficiency in homes through sensors, providing users with detailed information on energy consumption and suggestions for improving efficiency.

\subsection{Scope}
The system will include features for real-time data collection, energy consumption analysis, and reporting to users of high consumption.

\section{Overall Description}

\subsection{Product Perspective}
Ecometer will be an interactive web application that connects with sensors installed in homes to collect data on energy consumption. Data will be stored in a database for further analysis and presentations on dashboards.

\subsection{Product Features}

\begin{enumerate}
  \item \textbf{Registration}
     \begin{itemize}
        \item Users can register on the web application easily.
        \item Easy to use user interface for navigating.
     \end{itemize}

  \item \textbf{Data Collection}
     \begin{itemize}
        \item Sensors will collect real-time data on energy consumption in different areas of the home.
        \item Detailed data, such as consumption per device and per hour.
     \end{itemize}

  \item \textbf{Energy Consumption Analysis}
     \begin{itemize}
        \item The application will perform analysis on the collected data.
        \item The dashboards will show key data for decision making.
     \end{itemize}

  \item \textbf{Customized Reports}
     \begin{itemize}
        \item Generation of notifications if the consumption reaches high levels.
        \item Visual and easy-to-understand information, including charts and comparisons.
     \end{itemize}
\end{enumerate}

\subsection{User Classes and Characteristics}

\begin{itemize}
    \item \textbf{Residential Users:} Homeowners who wish to monitor and improve energy efficiency.
    \item \textbf{System Administrators:} Responsible for maintaining and updating the platform.
\end{itemize}

\section{Specific Requirements}

\subsection{Functional Requirements}

\subsubsection{Registration}

\begin{itemize}
    \item Users should be able to see new sensors associated with their account.
\end{itemize}

\subsubsection{Data Collection}

\begin{itemize}
    \item Sensors should send real-time data to the application.
    \item Secure storage of data in the system's database.
\end{itemize}

\subsubsection{Energy Consumption Analysis}

\begin{itemize}
    \item The application will perform statistical analysis on the collected data.
    \item Identification of consumption patterns and habits.
\end{itemize}

\subsubsection{Customized Reports}

\begin{itemize}
    \item Users can generate personalized notifications  on their energy consumption.
    \item Visual and easily understandable information, including charts and comparisons.
\end{itemize}

\subsection{Non-functional Requirements}

\subsubsection{Performance}

\begin{itemize}
    \item The application adapts to different devices.
    \item Response time for viewing reports should not exceed 5 seconds.
\end{itemize}

\subsubsection{Security}

\begin{itemize}
    \item All user and consumption data must be stored securely.
    \item The database will create backups daily in case of any security breach or data loss.
\end{itemize}

\section{Constraints}

\begin{itemize}
    \item The application must be compatible with modern browsers such as Chrome, Firefox, and Safari.
    \item Development will be carried out using web development technologies, such as Laravel and MYSQL.
\end{itemize}

\section{Approval}

The stakeholders involved in the project confirm that this Software Requirements Specification accurately reflects the needs and expectations of the users for the Ecometer project.

\end{document}
