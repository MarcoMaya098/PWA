\documentclass{article}
\usepackage[english]{babel}
\usepackage[utf8]{inputenc}
\usepackage{natbib}  % For bibliography styling
\usepackage{hyperref} % For clickable links

\begin{document}

\section*{Progressive Web Apps (PWA)}

PWA (Progressive Web App) is not a new term (Google introduced it in 2015), but it is unfamiliar to many people. PWAs are commonly defined as apps that bring together the best of web and native applications, even being seen as a middle ground or an evolved form.

\subsection*{Connectivity:}
\begin{itemize}
  \item Offline functionality: They can work even when the user is offline thanks to the ability to cache data.
  \item Progressive recovery: They offer a progressively enhanced experience based on the user's connection.
\end{itemize}

\subsection*{Responsive:}
\begin{itemize}
  \item Adaptive design: They adapt to different devices and screen sizes, providing a consistent user experience.
\end{itemize}

\subsection*{Navigability:}
\begin{itemize}
  \item Smooth navigation: They provide a seamless browsing experience without constantly reloading the page.
\end{itemize}

\subsection*{Installation:}
\begin{itemize}
  \item Home screen installation: Users can add the PWA to their home screen, just like with a native app.
\end{itemize}

\subsection*{Automatic Updates:}
\begin{itemize}
  \item Seamless updates: They are updated automatically, ensuring that users always have the latest version without the need to download or install manually.
\end{itemize}

\subsection*{Security:}
\begin{itemize}
  \item Secure connections: They run through HTTPS connections to ensure the security of transmitted information.
\end{itemize}

\subsection*{Discovery:}
\begin{itemize}
  \item Easy to find: They are indexable by search engines, making them easy to discover by users.
\end{itemize}

\subsection*{Direct Links:}
\begin{itemize}
  \item Shareable: They allow sharing direct links, facilitating distribution and promotion.
\end{itemize}

\subsection*{Interactivity:}
\begin{itemize}
  \item Fast interactivity: They offer an interactive and fast experience, similar to native applications.
\end{itemize}

\subsection*{Resource Consumption:}
\begin{itemize}
  \item Lower consumption: They use less storage space compared to native applications.
\end{itemize}

In summary, PWAs combine the best of both worlds, providing a rich and native-like user experience while leveraging the flexibility and accessibility of web applications.

\pagebreak
\section*{Differences between Web App, Service-Oriented App, and Progressive App}

\subsection*{Web App:}

\textbf{Definition:}
\begin{itemize}
  \item A web application is accessible through a web browser and runs entirely on the browser's platform.
\end{itemize}

\textbf{Advantages:}
\begin{itemize}
  \item Accessibility: Can be used on any device with a browser.
  \item Ease of development: Does not require installation and can be easily updated.
  \item Lower cost: Development and maintenance are usually more economical.
\end{itemize}

\textbf{Disadvantages:}
\begin{itemize}
  \item Connectivity dependence: Works better with an internet connection.
  \item Hardware access limitations: Limited access to device features such as the camera or sensors.
\end{itemize}

\subsection*{Service-Oriented App:}

\textbf{Definition:}
\begin{itemize}
  \item Focuses on integrating external services or APIs to provide specific functionalities.
\end{itemize}

\textbf{Advantages:}
\begin{itemize}
  \item Scalability: Can leverage existing external services.
  \item Specialized functionalities: Can offer specific features without developing everything from scratch.
  \item Independent updates: Services can be updated independently.
\end{itemize}

\textbf{Disadvantages:}
\begin{itemize}
  \item Integration complexity: Integrating and managing multiple services can be complicated.
  \item External dependence: The application may suffer if external services have issues.
\end{itemize}

\subsection*{Progressive App (PWA):}

\textbf{Definition:}
\begin{itemize}
  \item A PWA combines features of web and native applications to provide an advanced user experience.
\end{itemize}

\textbf{Advantages:}
\begin{itemize}
  \item Offline functionality: Can work offline thanks to caching.
  \item Native-like user experience: Smooth navigation and advanced features.
  \item Home screen installation: Can be installed as a native app on mobile devices.
\end{itemize}

\textbf{Disadvantages:}
\begin{itemize}
  \item Hardware access limitations: Although more advanced than a web app, it may still have restrictions compared to native applications.
  \item Requires browser support: Not all browsers offer full support for PWAs.
\end{itemize}

In summary, the choice between a web application, a service-oriented application, and a progressive application will depend on the specific requirements of the project, the necessary features, and development preferences. Each approach has its own advantages and disadvantages, and the choice will depend on development priorities and goals.
\pagebreak

\citep{latexproject,latexproject2,latexproject3,latexproject4}
\bibliographystyle{plainnat} % Choose a bibliography style compatible with natbib
\bibliography{references}     % Specify the .bib file with the extension

\end{document}
